% XeLaTeX atbalsts tiek pieslēgts ar šādām pakotnēm:
\usepackage{fontspec}
\usepackage{xunicode}
\usepackage{xltxtra}
\usepackage{indentfirst}	%pirmais indent katraa section, LOL

% Valodu atbalsts
\usepackage{polyglossia}
\setdefaultlanguage{latvian}
\setotherlanguages{english,russian}

% Lai sekcijas saktos ar 1
\renewcommand\thesection{\arabic{section}}

% Labojam saturu
\addto\captionslatvian{
	\renewcommand*\contentsname{\centering \MakeUppercase{Saturs}}
}

\renewcommand{\thefigure}{\arabic{section}.\arabic{figure}}

%attēli
\graphicspath{ {diagrams/}, {images/} }

%pielikumam
%\usepackage[toc,page]{appendix}

% Fonti -- var rakstīt sistēmas fontu nosaukumus
\setmainfont[Mapping=tex-text]{Times New Roman}
\setsansfont[Mapping=tex-text]{Times New Roman}
\newfontfamily\russianfont{Times New Roman}
\selectlanguage{latvian}

\title{Attēlu apstrāde lietojot heterogēnas, iegultās sistēmas}
\author{Rihards Novickis}

\input{titullapa.sty}